\documentclass[cjk,dvipdfmx,10pt,compress,t,fragile%
hyperref={bookmarks=true,bookmarksnumbered=true,bookmarksopen=false,%
colorlinks=false,%
pdftitle={第 134 回 関西 Debian 勉強会},%
pdfauthor={小林},%
%pdfinstitute={関西 Debian 勉強会},%
pdfsubject={資料},%
}]{beamer}

\title{DebianでRustパッケージング}
\author[Katsuki Kobayashi]{{\large\bf Katsuki Kobayashi}}
\institute[Debian JP]{{\normalsize\tt 関西 Debian 勉強会 共催}}
\date{{\small 2019年 3月 24日 (日)}}

\usepackage{graphicx}
\usepackage{moreverb}
\usepackage{ulem}
\usepackage[varg]{txfonts}
\usepackage{tabularx}
\usepackage{fancybox}
\usepackage{fancyvrb}
\usepackage{float}
\usepackage{multicol}
\usepackage{minijs}
\usepackage{amsmath}
\usepackage{amssymb}
\usepackage{newtxtext}
\usepackage{listings, listings-rust}
\usepackage{hyperref}
\AtBeginDvi{\special{pdf:tounicode EUC-UCS2}}
\usetheme{KansaiDebian}
\def\museincludegraphics{%
  \begingroup
  \catcode`\|=0
  \catcode`\\=12
  \catcode`\#=12
  \includegraphics[width=0.9\textwidth]}
%\renewcommand{\familydefault}{\sfdefault}
%\renewcommand{\kanjifamilydefault}{\sfdefault}
%
\newenvironment{commandline}%
{\VerbatimEnvironment
  \begin{Sbox}\begin{minipage}{0.9\hsize}\begin{fontsize}{8}{8} \color{white} \begin{BVerbatim}}%
{\end{BVerbatim}\end{fontsize}\end{minipage}\end{Sbox}
  \setlength{\fboxsep}{8pt}
% start on a new paragraph

\vspace{6pt}% skip before
\fcolorbox{white}{black}{\TheSbox}

\vspace{3pt}% skip after
}
%end of commandline

\begin{document}

\begin{frame}[fragile]
\titlepage
\end{frame}

\begin{frame}[fragile]{自己紹介}
 \begin{itemize}
  \item Katsuki Kobayashi
	\begin{itemize}
	 \item 組み込みエンジニア
	 \item 使用言語: C, アセンブラ(ARM)
	 \item \sout{永遠}に勉強中: Python, Kotlin, アセンブラ(RISC-V), Rust
	       \onslide<2>{\item 最近会社でC++を勉強させられている。タスケテ。}
	\end{itemize}
  \item Debianとのつきあい
	\begin{itemize}
	 \item 大学時代に素敵な先輩がた(うち現在DD2名,DM1名)によって布教
	 \item 個人的にはパッケージの構成の統一感が一番好き
	 \onslide<2>{\item \sout{あんまり貢献していない}}
	\end{itemize}
  \item Rust
	\begin{itemize}
	 \item 新しい言語が勉強したい
	 \item でもPythonやRubyはネイティブなバイナリにならなくてツラい
	 \onslide<2>{\item Goが流行ってるらしい}
	 \item じゃあMozilla好きだしRustにしよう \onslide<2>{←?}
	\end{itemize}
 \end{itemize}
\end{frame}

\begin{frame}[fragile]{本日の流れ}
\begin{itemize}
 \item Rustをさわってみる
       \begin{itemize}
	\item rustup
	\item Cargo
       \end{itemize}
 \item Rustの特徴のご紹介
       \begin{itemize}
	\item ownershipやlifetime
	\item パターンマッチング
	\item IteratorとClosures
	\item Trait
       \end{itemize}
 \item Debianパッケージング
\end{itemize}
\end{frame}

\begin{frame}[fragile]{Rustとは?}
 \begin{itemize}
  \item Mozillaが開発したシステムプログラミング言語
	\begin{itemize}
	 \item Servo(絶賛開発中))というブラウザエンジンのために開発
	 \item 安全性・並行性について考えられて設計されている
	 \item 静的型付け言語
	\end{itemize}
  \item カニ?
	\begin{itemize}
	 \item Rustプログラマーの事をRustacean(ラストシアン)と呼ぶ
	 \item ``Crustacean(甲殻類)`` から来ているらしい
	       \begin{itemize}
		\item オライリー本の表紙はオオヒロバオウギガニ
		\item 非公式なマスコットもカニ \href{http://rustacean.net/}{(Ferrisっていう模様)}
	       \end{itemize}
	\end{itemize}
 \end{itemize}
\end{frame}

\begin{frame}[fragile]{インストール}
 \begin{itemize}
  \item \texttt{rustup}を入れるのが一応公式
	\begin{itemize}
	 \item \verb@curl https://sh.rustup.rs -sSf | sh@
	 \item あなたのhome dirに色々と入ります \verb@~/.cargo@
	\end{itemize}
  \item \texttt{rustup}のサブコマンドたち
	\begin{description}
	 \item[default] デフォルトのツールチェーンを切り替えます。
		    \begin{itemize}
		     \item ツールチェーン: stableかnightlyか特定のバージョンを指定できる
		    \end{itemize}
	 \item[update] 更新をかけます。
		    nightlyを使う場合はちょくちょく使う
		    \begin{itemize}
		     \item そしてたまに壊れる
		     \item 壊れるといっても、コンパイラがおかしくなった事はないです
		    \end{itemize}
	 \item[completions] シェルの補完用のコードを吐く (bash, zsh, fish, \textbf{PowerShell}, etc..)
	\end{description}
 \end{itemize}
\end{frame}

\begin{frame}[fragile]{Debianパッケージでのインストール}
 \begin{itemize}
  \item もちろんDebianのパッケージもある
	\begin{itemize}
	 \item ビルドツールであるcargoのパッケージを入れるのがよろしいかと
	 \item 一緒にコンパイラ(rustc)も入ります
	\end{itemize}
 \end{itemize}
 \begin{commandline}
% apt show cargo
Package: cargo
Version: 0.33.0-1
Priority: optional
Section: rust
Maintainer: Rust Maintainers <pkg-rust-maintainers@alioth-lists.debian.net>
Installed-Size: 9,453 kB
Depends: libc6 (>= 2.18), libcurl3-gnutls (>= 7.28.0), libgcc1 (>= 1:4.2), \
libgit2-27 (>= 0.26.0), libssh2-1 (>= 1.2.5), libssl1.1 (>= 1.1.0), \
zlib1g (>= 1:1.1.4), rustc (>= 1.24), binutils, gcc | clang | c-compiler
Suggests: cargo-doc, python3
Homepage: https://crates.io/
Download-Size: 2,483 kB
APT-Manual-Installed: yes
APT-Sources: http://ftp.jp.debian.org/debian sid/main amd64 Packages
 \end{commandline}
\end{frame}

\begin{frame}[fragile]{Hello World (1/)}
 \begin{itemize}
  \item Rustのプロジェクトは、 \verb|cargo new| で作ります
	\begin{itemize}
	 \item \verb|--bin| (現default) か \verb|--lib| (旧default) でパッケージの種類も指定できます
	\end{itemize}
 \end{itemize}
\begin{commandline}
% cargo new hello
     Created binary (application) `hello` package
\end{commandline}
\begin{itemize}
 \item 実行すると、Cargo.tomlとsrc/main.rsができる
\end{itemize}
\begin{commandline}
% tree
.
├── Cargo.toml
└── src
    └── main.rs

1 directory, 2 files
\end{commandline}
\end{frame}

\begin{frame}[fragile]{Hello World (2/)}
 \begin{itemize}
  \item 実はプロジェクトを作った時点でHello Worldの半分が完了している
	\begin{itemize}
	 \item 生成されたsrc/main.rsの中身↓
	\end{itemize}
 \end{itemize}
\begin{lstlisting}[language=Rust,style=boxed,style=colouredRust]
fn main() {
    println!("Hello, world!");
}\end{lstlisting}
\begin{itemize}
 \item 実行するには \verb|cargo run|
\end{itemize}
\begin{commandline}
% cargo run
   Compiling hello v0.1.0 (/path/to/hello)
    Finished dev [unoptimized + debuginfo] target(s) in 0.31s
     Running `target/debug/hello`
Hello, world!
\end{commandline}
\end{frame}

\begin{frame}[fragile]{Hello World (3/)}
\begin{lstlisting}[language=Rust,style=boxed,style=colouredRust]
fn main() {
    println!("Hello, world!");
}\end{lstlisting}
\begin{itemize}
 \item 少しコードを見てみる
       \begin{description}
	\item[関数定義] \verb|fn|
       \end{description}
\end{itemize}
\end{frame}

\takahashi[50]{\vspace*{2zw} 以上}

\begin{frame}[fragile]{どっかに挿入予定のメモ}
 \begin{itemize}
  \item Rustのファイルの拡張子は .rs
	\begin{itemize}
	 \item Rustで書かれたプロジェクトの公式サイトが \verb|http://xxx.rs| であることが多い
	 \item 国別コードトップドメイン .rs : セルビア
	\end{itemize}
  \item \texttt{rust-fmt}というツールがある
	\begin{itemize}
	 \item \texttt{gofmt}とか\texttt{autopep8}とか\texttt{clang-format}的なツール
	\end{itemize}
  \item Podcast (Rebuild.fm?) で誰かが言ってた
	\begin{itemize}
	 \item Go: C++が嫌いで駆逐したい
	 \item Rust: C++が大好きだけど辛いのでなんとかしたい
	\end{itemize}
  \item Rustacean
	\begin{itemize}
	 \item LISPer, Pythonista, Rubyist, Gopher, TeXnician的な
	 \item ``crustaceans'' : 甲殻類
	       \begin{itemize}
		\item オライリー本の表紙はオオヒロバオウギガニ
		\item 非公式なマスコットもカニ \href{http://rustacean.net/}{(Ferrisっていう模様)}
	       \end{itemize}
	\end{itemize}
 \end{itemize}
\end{frame}

\begin{frame}[fragile]{どっかに挿入予定のメモ}
 \begin{itemize}
  \item とりあえずパッケージにするには
	\begin{itemize}
	 \item \verb|dh_make| を実行
	 \item debian/rulesの\texttt{dh}の引数に\verb|--buildsystem cargo|を追加
	\end{itemize}
 \end{itemize}
 \begin{commandline}
cp: './debian/cargo-checksum.json' を stat できません: \
そのようなファイルやディレクトリはありません
 \end{commandline}
 \begin{itemize}
  \item むぅ?
	\begin{itemize}
	 \item \texttt{cargo-vendor}で各crateのchecksumは作れるけど、
	       \texttt{main.rs}なアプリはどうしたら?
	\end{itemize}
 \end{itemize}
\end{frame}

\begin{frame}[fragile]{どっかに挿入予定のメモ}
 \begin{itemize}
  \item \texttt{debcargo}, \texttt{debcargo-conf}
	\begin{itemize}
	 \item crate.ioに存在するcrateにしか対応していない模様
	\end{itemize}
 \end{itemize}
\end{frame}

\end{document}