\documentclass[cjk,14pt,xcolor=dvipsnames,table,dvipdfmx,professional font,t,fragile]{beamer}
\usetheme{Copenhagen}
\usepackage{ulem}
\usepackage{tabularx}
\usepackage{fancybox}
\usepackage{fancyvrb}
\usepackage{float}
\usepackage{multicol}
\usepackage{minijs}
\usepackage{amsmath}
\usepackage{amssymb}
%\usepackage{minted}
\usepackage{ulem}
\usepackage{newtxtext}
\usepackage{listings, listings-rust}
\usepackage{hyperref}


\renewcommand{\familydefault}{\sfdefault}
\renewcommand{\kanjifamilydefault}{\gtdefault}
%
\hypersetup{
 pdfauthor={小林 克希},
 pdftitle={RustBook勉強会},
 pdfkeywords={Rust},
 pdfsubject={},
 pdfcreator={pLaTeX + dvipdfmx},
 pdflang={Japanese}
}
%
\newenvironment{commandline}%
{\VerbatimEnvironment
  \begin{Sbox}\begin{minipage}{0.9\hsize}\begin{fontsize}{8}{8} \color{white} \begin{BVerbatim}}%
{\end{BVerbatim}\end{fontsize}\end{minipage}\end{Sbox}
  \setlength{\fboxsep}{8pt}
% start on a new paragraph

\vspace{6pt}% skip before
\fcolorbox{white}{black}{\TheSbox}

\vspace{3pt}% skip after
}
%end of commandlinesmall

\begin{document}
\title{RustBook勉強会}
\subtitle{11. Testing}
\author{Katsuki Kobayashi}
\date{2019-02-23}

\maketitle

\begin{frame}{予めお詫び}
 \begin{itemize}
  \item 発表者はC言語/ARMアセンブラのみ書ける
  \item ということで、他の言語とかさっぱりです
	\begin{itemize}
	 \item 「あの言語では〜〜」とかいうのは\\
	       全体に聞いてください
	 \item 不幸にも最近C++の本を読まされています\\
	       タスケテ
	\end{itemize}
  \item 社風:「テスト? なにそれ食べられるの?」
  \item 「世間一般ではこんな感じ」的なツッコミ歓迎
 \end{itemize}
\end{frame}

\begin{frame}{自動テストを書こう!}
 \begin{itemize}
  \item ダイクストラ先生曰く(1972年のエッセイ)
	\begin{itemize}
	 \item 	{\scriptsize
		``Program testing can be very effective way to show the presence of bugs,
		but it is hopelessly inadequate for showing their absence.''}
	 \item バグの存在を示す効果的な方法ではあるが、\\
	       無い事を示すには絶望的に不充分である
	\end{itemize}
  \item でも、「テストを頑張るべきではない」、\\
	という意味ではない
 \end{itemize}
\end{frame}

\begin{frame}{正確さ(correctness)}
 \begin{itemize}
  \item Rustは正確さについてかなり考えられている
	\begin{itemize}
	 \item が、やっぱり正確であることの証明は大変
	\end{itemize}
  \item Rustの型システムはこの重役を担っている
	\begin{itemize}
	 \item が、型だけで不正確を全部抑えられない
	\end{itemize}
  \item というわけで、Rustは言語レベルでの\\
	テストの自動化をサポートします
 \end{itemize}
\end{frame}

\begin{frame}{例: \texttt{add\_two()} (1/2)}
 \begin{itemize}
  \item \texttt{add\_two()}
	\begin{itemize}
	 \item 入力された値に2を加えて返す関数
	 \item 入出力はともに整数
	\end{itemize}
  \item コンパイルするとRustがチェック
	\begin{itemize}
	 \item 型の一致 (Stringを入れたらエラー)
	 \item borrow (変な参照をしたらエラー)
	\end{itemize}
  \item 関数の挙動はノーチェック
	\begin{itemize}
	 \item $10$ 足されようが $50$ 引かれようが\\
	       エラーにはならない
	\end{itemize}
  \item 「自動テスト」の出番
 \end{itemize}
\end{frame}

\begin{frame}{例: \texttt{add\_two()} (2/2)}
 \begin{itemize}
  \item テストの方法の例
	\begin{itemize}
	 \item $3$ を与えたら $5$ を返すとassertをする
	 \item このテストを、コードの変更の度に行なう
	 \item 正しい挙動をしている事を確認
	\end{itemize}
 \end{itemize}
\end{frame}

\begin{frame}{このチャプターは}
 \begin{itemize}
  \item テストは難しいので
	\begin{itemize}
	 \item 良いテストの詳細についてはカバーしない
	 \item Rustののテストのための機能について議論する
	\end{itemize}
  \item 取り扱う内容
	\begin{itemize}
	 \item アノテーション
	 \item マクロ
	 \item デフォルトの挙動
	 \item オプション
	 \item unit testとintegration testの構成方法
	\end{itemize}
 \end{itemize}
\end{frame}

\begin{frame}{How to Write Tests}
 \begin{itemize}
  \item Rustのテスト
	\begin{itemize}
	 \item コードが期待通り動くかを確認するテスト関数
	\end{itemize}
  \item テスト関数では一般に以下の3つのアクションを実行
	\begin{itemize}
	 \item 必要なデータや状態を準備する
	 \item テストしたいコードを実行する
	 \item 結果が期待通りかアサートする
	\end{itemize}
  \item 以降、Rustがテストのために提供しているものを紹介
	\begin{itemize}
	 \item \texttt{test} attribute
	 \item マクロ
	 \item \texttt{should\_panic} attribute、等
	\end{itemize}
 \end{itemize}
\end{frame}

\begin{frame}[containsverbatim]{The Anatomy of a Test Function(1/7)}
 \begin{itemize}
  \item ぶっちゃけ、Rustのテストとは\\
	\texttt{test} attribute をつけた関数のこと
	\begin{itemize}
	 \item attribute: Chapt. 5で使った \texttt{derive} とか
	\end{itemize}
  \item test attributeの付け方: \\
	\hspace{2zw} \texttt{fn} の前に \verb|#[test]| を付ける
  \item \texttt{test} attributeを付けると
	\begin{itemize}
	 \item \texttt{cargo test}コマンド
	 \item テストランナーをビルドして実行
	 \item レポートを表示 (passes or fails)
	\end{itemize}
 \end{itemize}
\end{frame}

\begin{frame}[containsverbatim]{The Anatomy of a Test Function(2/7)}
 \begin{itemize}
  \item \verb|cargo new adder --lib| する
 \end{itemize}
 \begin{commandline}
% cargo new adder --lib
      Created library `adder` package
 \end{commandline}
 \begin{itemize}
  \item \texttt{src/libs.rs} が以下の内容で作成される
 \end{itemize}
 {\scriptsize
 \begin{lstlisting}[language=Rust,style=boxed,style=colouredRust]
#[cfg(test)]
mod tests {
    #[test]
    fn it_works() {
        assert_eq!(2 + 2, 4);
    }
}\end{lstlisting}}
 \begin{itemize}
  \item とりあえず \verb|#[cfg(test)]| と \verb|mod tests| の行は無視
 \end{itemize}
\end{frame}

\begin{frame}[containsverbatim]{The Anatomy of a Test Function(3/7)}
 \mbox{}
 {\scriptsize
 \begin{lstlisting}[language=Rust,style=boxed,style=colouredRust]
#[cfg(test)]
mod tests {
    #[test]
    fn it_works() {
        assert_eq!(2 + 2, 4);
    }
}\end{lstlisting}}
 \begin{itemize}
  \item \verb|#[test]|
	\begin{itemize}
	 \item テスト関数を示す
	 \item 共通処理等を書く非テスト関数も書けるように
	\end{itemize}
  \item \verb|assert_eq!(2 + 2, 4)|
	\begin{itemize}
	 \item $2 + 2$ が $4$ である事をアサート
	\end{itemize}
 \end{itemize}
\end{frame}

\begin{frame}[containsverbatim]{The Anatomy of a Test Function(4/7)}
 \begin{commandline}
% cargo test
   Compiling adder v0.1.0 (/home/rare/work/slides/RustBook/samples/adder)
    Finished dev [unoptimized + debuginfo] target(s) in 0.53s
     Running target/debug/deps/adder-1d0dfd1494bab5c6

running 1 test
test tests::it_works ... ok

test result: ok. 1 passed; 0 failed; 0 ignored; 0 measured; 0 filtered out

   Doc-tests adder

running 0 tests

test result: ok. 0 passed; 0 failed; 0 ignored; 0 measured; 0 filtered out
 \end{commandline}
 \begin{itemize}
  \item \href{https://play.integer32.com/?version=stable&mode=debug&edition=2018&gist=ce8ee04d2dcc2311a14efc8090ebda51}
	{Playground} でもテストできる……
  \item ignored, filteredについては次節で説明
  \item measuredについては現在はnightlyだけの機能
	\begin{itemize}
	 \item ベンチマークテスト用
	\end{itemize}
 \end{itemize}
\end{frame}

\begin{frame}[containsverbatim]{The Anatomy of a Test Function(5/7)}
 \begin{itemize}
  \item \verb|Doc-tests adder| という出力がある
	\begin{itemize}
	 \item ドキュメントテストの結果
	 \item APIのドキュメントのコーディング例もコンパイル/テスト可能
	 \item ドキュメントとコードの同期に役立つ!!
	\end{itemize}
  \item Chapt. 14 ``Documentation Comments'' でやるのでスルーします
 \end{itemize}
\end{frame}

\begin{frame}[containsverbatim]{The Anatomy of a Test Function(6/7)}
 \begin{itemize}
  \item テストの名前を変更
	\begin{itemize}
	 \item \verb|it_works()| を \verb|exploration()| にしてみる
	 \item  \href{https://play.integer32.com/?version=stable&mode=debug&edition=2018&gist=dc15fce8e988e1efd0a2dd8d8d87dd27}
		{Playground}
	 \item 無事に結果の表示に反映
	\end{itemize}
 \end{itemize}
 \begin{commandline}
running 1 test
test tests::exploration ... ok
 \end{commandline}
\end{frame}

\begin{frame}[containsverbatim]{The Anatomy of a Test Function(7/7)}
 \begin{itemize}
  \item 失敗するテストを入れてみる
	\begin{itemize}
	 \item \verb|panic!| マクロを使って \verb|another()| 関数を作る
	 \item \href{https://play.integer32.com/?version=stable&mode=debug&edition=2018&gist=71238afa6168ec8b26d4bf761d6404a6}
	       {Playground}
	\end{itemize}
 \end{itemize}
 {\scriptsize
 \begin{lstlisting}[language=Rust,style=boxed,style=colouredRust]
    #[test]
    fn another() {
        panic!("Make this test fail");
    }\end{lstlisting}}
 \begin{itemize}
  \item 無事失敗する (前回より2つ項目が増えてる)
 \end{itemize}
 \vspace*{-10pt}
 \begin{commandline}
test tests::another ... FAILED

failures:

---- tests::another stdout ----
thread 'tests::another' panicked at 'Make this test fail!', src/lib.rs:10:9

failures:
    tests::another
 \end{commandline}
\end{frame}

\begin{frame}[containsverbatim]{Checking Results with the assert! (1/5)}
 \begin{itemize}
  \item \verb|assert!| マクロ
	\begin{itemize}
	 \item true: 何もしない
	 \item false: \verb|panic!| をコール
	\end{itemize}
  \item Chapt. 5の \texttt{Rectanble}構造体を例に
	\begin{itemize}
	 \item 	\href{https://play.integer32.com/?version=stable&mode=debug&edition=2018&gist=dc15fce8e988e1efd0a2dd8d8d87dd27}
		{Playground}
	\end{itemize}
 \end{itemize}
 {\scriptsize
 \begin{lstlisting}[language=Rust,style=boxed,style=colouredRust]
#[derive(Debug)]
pub struct Rectangle {
    length: u32,
    width: u32,
}

impl Rectangle {
    pub fn can_hold(&self, other: &Rectangle) -> bool {
        self.length > other.length && self.width > other.width
    }
}\end{lstlisting}}

\end{frame}

\begin{frame}[containsverbatim]{Checking Results with the assert! (2/5)}
\mbox{}
 {\scriptsize
 \begin{lstlisting}[language=Rust,style=boxed,style=colouredRust]
#[derive(Debug)]
pub struct Rectangle {
    length: u32,
    width: u32,
}

impl Rectangle {
    pub fn can_hold(&self, other: &Rectangle) -> bool {
        self.length > other.length && self.width > other.width
    }
}\end{lstlisting}}
 \begin{itemize}
   \item \verb|can_hold()| メソッド
	 \begin{itemize}
	  \item ブール値を返す\\
		\hspace{3zw} $\rightarrow$ \verb|assert!| でのテストに最適!!
	  \item \verb|length| が $8$ , \verb|width| が $7$ の \verb|Rectangle| が\\
		\verb|length| が $5$ , \verb|width| が $1$ の \verb|Rectangle|
		をholdできる事をテスト
	 \end{itemize}
 \end{itemize}
\end{frame}

\begin{frame}[containsverbatim]{Checking Results with the assert! (3/5)}
 \begin{itemize}
  \item \verb|larger_can_hold_smaller()| テスト
	\begin{itemize}
	 \item \href{https://play.integer32.com/?version=stable&mode=debug&edition=2018&gist=cdf6bb94826b7c2d369e3419ac251cac}
	       {Playground} (\verb|use super::*;| する必要あり)
	\end{itemize}
 \end{itemize}
 {\scriptsize
 \begin{lstlisting}[language=Rust,style=boxed,style=colouredRust]
    #[test]
    fn larger_can_hold_smaller() {
        let larger = Rectangle { length: 8, width: 7 };
        let smaller = Rectangle { length: 5, width: 1 };

        assert!(larger.can_hold(&smaller));
    }\end{lstlisting}}
 \begin{commandline}
running 1 test
test tests::larger_can_hold_smaller ... ok

test result: ok. 1 passed; 0 failed; 0 ignored; 0 measured; 0 filtered out
 \end{commandline}
 \begin{itemize}
  \item \verb|can_hold(0)| は \texttt{true} を返すのでpassする
 \end{itemize}
\end{frame}

\begin{frame}[containsverbatim]{Checking Results with the assert! (4/5)}
 \begin{itemize}
  \item \verb|smaller_cannot_hold_larger()| テスト
	\begin{itemize}
	 \item \href{https://play.integer32.com/?version=stable&mode=debug&edition=2018&gist=6eff4f70eba2b909a0cd377e1ebad141}
	       {Playground}
	\end{itemize}
 \end{itemize}
 {\scriptsize
 \begin{lstlisting}[language=Rust,style=boxed,style=colouredRust]
    #[test]
    fn smaller_cannot_hold_larger() {
        let larger = Rectangle { length: 8, width: 7 };
        let smaller = Rectangle { length: 5, width: 1 };

        assert!(!smaller.can_hold(&larger));  // '!' がミソ
    }\end{lstlisting}}
 \begin{commandline}
running 2 tests
test tests::larger_can_hold_smaller ... ok
test tests::smaller_cannot_hold_larger ... ok

test result: ok. 2 passed; 0 failed; 0 ignored; 0 measured; 0 filtered out
 \end{commandline}
 \begin{itemize}
  \item \verb|smaller_cannot_hold_larger()| が \verb|false| を返すのでpassする
 \end{itemize}
\end{frame}

\begin{frame}[containsverbatim]{Checking Results with the assert! (5/5)}
 \begin{itemize}
  \item バグを仕込んでみる
	\begin{itemize}
	 \item \texttt{length} の不等号が逆
	 \item \href{https://play.integer32.com/?version=stable&mode=debug&edition=2018&gist=67c1f9b79733a2ebeec35d8556b00ca4}
	       {Playground}
	\end{itemize}
 \end{itemize}
 {\scriptsize
 \begin{lstlisting}[language=Rust,style=boxed,style=colouredRust]
impl Rectangle {
    pub fn can_hold(&self, other: &Rectangle) -> bool {
        self.length < other.length && self.width > other.width
    } //            ^
}\end{lstlisting}}
\begin{itemize}
 \item 目論見通りfailする
\end{itemize}
\begin{commandline}
running 2 tests
test tests::larger_can_hold_smaller ... FAILED
test tests::smaller_cannot_hold_larger ... ok

failures:

---- tests::larger_can_hold_smaller stdout ----
thread 'tests::larger_can_hold_smaller' panicked at 'assertion failed: larger.can_hold(&smaller)', src/lib.rs:29:9
note: Run with `RUST_BACKTRACE=1` for a backtrace.
\end{commandline}
\end{frame}

\begin{frame}[containsverbatim]{Testing Equality with the assert\_eq! (1/4)}
 \begin{itemize}
  \item テストの一般的な方法は期待値と結果の比較
	\begin{itemize}
	 \item \verb|assert!| マクロに $==$ を渡してもよい
	 \item \verb|assert_eq!|,  \verb|assert_ne!| マクロもある
	 \item これらは期待値と結果の両方を表示してくれる
	\end{itemize}
 \end{itemize}
\end{frame}

\begin{frame}[containsverbatim]{Testing Equality with the assert\_eq! (2/4)}
 \begin{itemize}
  \item \verb|add_two()| による例:
	\href{https://play.integer32.com/?version=stable&mode=debug&edition=2018&gist=3e41a536cc2029885d833562d24c9a30}
	{(Playground)}
 \end{itemize}
 {\scriptsize
 \begin{lstlisting}[language=Rust,style=boxed,style=colouredRust]
fn main() {}
pub fn add_two(a: i32) -> i32 {
    a + 2
}

#[cfg(test)]
mod tests {
    use super::*;

    #[test]
    fn it_adds_two() {
        assert_eq!(4, add_two(2));
    }
}\end{lstlisting}}
\begin{commandline}
test tests::it_adds_two ... ok
\end{commandline}
\end{frame}

\begin{frame}[containsverbatim]{Testing Equality with the assert\_eq! (3/4)}
 \begin{itemize}
  \item 失敗するようにする
	\href{https://play.integer32.com/?version=stable&mode=debug&edition=2018&gist=dfd0a140bf3dcd3168db3debdf3e064b}
	{(Playground)}
 \end{itemize}

 {\scriptsize
 \begin{lstlisting}[language=Rust,style=boxed,style=colouredRust]
fn main() {}
pub fn add_two(a: i32) -> i32 {
    a + 3   // なぜか3 を足す
}\end{lstlisting}}
\begin{commandline}
thread 'tests::it_adds_two' panicked at 'assertion failed: `(left == right)`
  left: `4`,
 right: `5`', src/lib.rs:11:9
\end{commandline}
\begin{itemize}
 \item \texttt{left}と\texttt{right}が表示される
       \begin{itemize}
	\item 他の言語やテストフレームワークだと
	      \texttt{expected} と \texttt{actual} と呼んだりする
	\item Rustは期待値と結果の順番を気にしないので
	      \verb|assert_eq!(add_two(2), 4);| としてもOK
       \end{itemize}
\end{itemize}
\end{frame}

\begin{frame}[containsverbatim]{Testing Equality with the assert\_eq! (4/4)}
 \begin{itemize}
  \item \verb|assert_ne!|マクロもある
	\begin{itemize}
	 \item $2$つの値が異なればpass、等しければfail
	 \item 「結果が正確に解らないが特定の値にはならない」
	       というテストに使う
	 \item 例えば、結果が実行する時間によって変化するもの
	\end{itemize}
  \item \verb|assert_eq!|と\verb|assert_ne!|
	\begin{itemize}
	 \item 使用する値には PartialEq と Debugトレイトが実装されている必要がある
	       \begin{itemize}
		\item '$==$' と '$!=$' で比較
		\item 結果をデバッグフォーマットで表示する
	       \end{itemize}
	 \item 独自に実装した構造体やenumについては
	       \verb|#[derive(PartialEq, Debug)]| と付ければ大抵OK
	\end{itemize}
 \end{itemize}
\end{frame}

\begin{frame}[containsverbatim]{Adding Custom Failure Messages(1/5)}
 \begin{itemize}
  \item 失敗時のメッセージを追加できる
	\begin{itemize}
	 \item \verb|assert!|, \verb|assert_eq!|, \verb|assert_ne!| マクロの
	       引数の続きに \verb|format!| マクロに渡す引数を付ける
	 \item ``\verb|{}| textholder''を書いて値を表示できる
	\end{itemize}
 \end{itemize}
\end{frame}

\begin{frame}[containsverbatim]{Adding Custom Failure Messages(2/5)}
 \begin{itemize}
  \item greet関数による例
	\begin{itemize}
	 \item (\href{https://play.integer32.com/?version=stable&mode=debug&edition=2018&gist=0253d4f4ae81c9de7103a3453bd8290b}
	       {Playground})
	\end{itemize}
 \end{itemize}
 {\scriptsize
 \begin{lstlisting}[language=Rust,style=boxed,style=colouredRust]
pub fn greeting(name: &str) -> String {
    format!("Hello {}!", name)
}

#[cfg(test)]
mod tests {
    use super::*;

    #[test]
    fn greeting_contains_name() {
        let result = greeting("Carol");
        assert!(result.contains("Carol"));
    }
}
\end{lstlisting}}
 \begin{itemize}
  \item 今後の仕様変化への対応のため、出力に引数(名前)が含まれているかでテストしている
 \end{itemize}
\end{frame}

\begin{frame}[containsverbatim]{Adding Custom Failure Messages(3/5)}
 \begin{itemize}
  \item 当然、テストは通る
 \end{itemize}
\begin{commandline}
running 1 test
test tests::greeting_contains_name ... ok

test result: ok. 1 passed; 0 failed; 0 ignored; 0 measured; 0 filtered out
\end{commandline}
 \begin{itemize}
  \item では、壊しましょう
	\begin{itemize}
	 \item \href{https://play.integer32.com/?version=stable&mode=debug&edition=2018&gist=d37ff552d3017ee02fcc5f8e1eb1bcef}
	       {(Playground)}
	\end{itemize}
 \end{itemize}
 {\scriptsize
 \begin{lstlisting}[language=Rust,style=boxed,style=colouredRust]
pub fn greeting(_name: &str) -> String {
    String::from("Hello!") // 引数を含まないように
}
\end{lstlisting}}
\end{frame}

\begin{frame}[containsverbatim]{Adding Custom Failure Messages(4/5)}
 \begin{itemize}
  \item 壊れました
 \end{itemize}
 \begin{commandline}
failures:

---- tests::greeting_contains_name stdout ----
thread 'tests::greeting_contains_name' panicked at \
'assertion failed: result.contains("Carol")', src/lib.rs:12:9
note: Run with `RUST_BACKTRACE=1` for a backtrace.
 \end{commandline}
\begin{itemize}
 \item でも
       \begin{itemize}
	\item テストが失敗した\texttt{assert}の内容だけ表示される
	      \begin{itemize}
	       \item \verb|result.contains("Carol")|
	      \end{itemize}
	\item \verb|greeting()| 関数の出力を表示した方が便利
       \end{itemize}
\end{itemize}
\end{frame}

\begin{frame}[containsverbatim]{Adding Custom Failure Messages(5/5)}
 \begin{itemize}
  \item テストを変更する
	\begin{itemize}
	 \item \href{https://play.integer32.com/?version=stable&mode=debug&edition=2018&gist=1a0c93fc1cad391ba20a1a70a80efff7}
	       {(Playground)}
	\end{itemize}
 \end{itemize}
 {\scriptsize
 \begin{lstlisting}[language=Rust,style=boxed,style=colouredRust]
#[test]
fn greeting_contains_name() {
    let result = greeting("Carol");
    assert!(
        result.contains("Carol"),
        "Greeting did not contain name, value was `{}`", result
    );
}\end{lstlisting}}
 \begin{commandline}
failures:

---- tests::greeting_contains_name stdout ----
thread 'tests::greeting_contains_name' panicked at \
'Greeting did not contain name, value was `Hello!`', src/lib.rs:12:9
note: Run with `RUST_BACKTRACE=1` for a backtrace.
 \end{commandline}
\end{frame}

\begin{frame}[containsverbatim]{Panic with \texttt{should\_panic} (1/8)}
 \begin{itemize}
  \item 結果が正しいかのチェックだけでなく、
	エラー条件についてのチェックも重要
  \item 例として \texttt{Guess} type(9章)で説明します
	\begin{itemize}
	 \item 構造体のvalueフィールド: $1 \sim 100$
	 \item \verb|new()|メソッドで値が範囲外なpanicする
	\end{itemize}
 \end{itemize}
 {\scriptsize \begin{lstlisting}[language=Rust,style=boxed,style=colouredRust]
pub struct Guess {
    value: i32,
}
impl Guess {
    pub fn new(value: i32) -> Guess {
        if value < 1 || value > 100 {
            panic!("Guess value must be between 1 and 100, got {}.", value);
        }
        Guess {
            value
        }
    }
}\end{lstlisting}}
\end{frame}

\begin{frame}[containsverbatim]{Panic with \texttt{should\_panic} (2/8)}
 \begin{itemize}
  \item 範囲外の値を入れてpanicさせるテストを書く
	\begin{itemize}
	 \item attributeを追加する : \verb|should_panic|
	 \item panicしたらpass, しなかったらfail
	 \item \href{https://play.rust-lang.org/?version=stable&mode=debug&edition=2018&gist=d997d3b4a728cac8457b9bbc650e78bf}
	       {(Playground)}
	\end{itemize}
 \end{itemize}
 \vspace{-.5zw}
 {\scriptsize
\begin{lstlisting}[language=Rust,style=boxed,style=colouredRust]
#[cfg(test)]
mod tests {
    use super::*;

    #[test]
    #[should_panic]  // 順番は逆でも良い模様
    fn greater_than_100() {
        Guess::new(200);
    }
}\end{lstlisting}}
 \vspace{-.5zw}
\begin{commandline}
running 1 test
test tests::greater_than_100 ... ok

test result: ok. 1 passed; 0 failed; 0 ignored; 0 measured; 0 filtered out
\end{commandline}
\end{frame}

\begin{frame}[containsverbatim]{Panic with \texttt{should\_panic} (3/8)}
 \begin{itemize}
  \item バグを仕込みましょう
	\begin{itemize}
	 \item \href{https://play.rust-lang.org/?version=stable&mode=debug&edition=2018&gist=69a9efedbdf8007e106163ab5236cf52}
	       {(Playground)}
	\end{itemize}
 \end{itemize}
 {\scriptsize
\begin{lstlisting}[language=Rust,style=boxed,style=colouredRust]
impl Guess {
    pub fn new(value: i32) -> Guess {
        if value < 1 /* 100より大きい条件を消す */ {
            panic!("Guess value must be between 1 and 100, got {}.", value);
        }
        Guess {
            value
        }
    }
}\end{lstlisting}}
\begin{itemize}
 \item 無事に失敗する
\end{itemize}
\begin{commandline}
running 1 test
test tests::greater_than_100 ... FAILED
\end{commandline}
\end{frame}

\begin{frame}[containsverbatim]{Panic with \texttt{should\_panic} (4/8)}
 \begin{commandline}
running 1 test
test tests::greater_than_100 ... FAILED
 \end{commandline}
\begin{itemize}
 \item メッセージが不親切
 \item 加えて、\verb|should_panic|は不正確になりがち
       \begin{itemize}
	\item テストが期待とは異なるpanicを起こしてもpassする
       \end{itemize}
 \item もうちょっと親切かつ正確にするために \verb|expected| パラメーターを
       \verb|should_panic| attributeに追加する
\end{itemize}
\end{frame}

\begin{frame}[containsverbatim]{Panic with \texttt{should\_panic} (5/8)}
 \begin{itemize}
  \item \verb|new()|メソッドを変更
	\begin{itemize}
	 \item 初期値が大きすぎるか、
	       小さすぎるかで、panic時のメッセージを変更する
	\end{itemize}
 \end{itemize}
 {\scriptsize
\begin{lstlisting}[language=Rust,style=boxed,style=colouredRust]
pub fn new(value: i32) -> Guess {
    if value < 1 {
        panic!("Guess value must be greater than or equal to 1, got {}.",
               value);
    } else if value > 100 {
        panic!("Guess value must be less than or equal to 100, got {}.",
               value);
    }
}\end{lstlisting}}
\end{frame}

\begin{frame}[containsverbatim]{Panic with \texttt{should\_panic} (6/8)}
 \begin{itemize}
  \item \verb|should_panic| に \verb|expected| パラメータ
	\begin{itemize}
	 \item \href{https://play.rust-lang.org/?version=stable&mode=debug&edition=2018&gist=498dda42617ff9de462a2edb7ef82a05}
	       {(Playground)}
	 \item panic時のメッセージのsubstringを指定する
	\end{itemize}
 \end{itemize}
 {\scriptsize
\begin{lstlisting}[language=Rust,style=boxed,style=colouredRust]
#[cfg(test)]
mod tests {
    use super::*;

    #[test]
    #[should_panic(
        expected = "Guess value must be less than or equal to 100")]
    fn greater_than_100() {
        Guess::new(200);
    }
}\end{lstlisting}}
\begin{commandline}
running 1 test
test tests::greater_than_100 ... ok
\end{commandline}
\end{frame}

\begin{frame}[containsverbatim]{Panic with \texttt{should\_panic} (7/8)}
 \begin{itemize}
  \item バグを仕込む
	\begin{itemize}
	 \item panicの文言を入れ替えた
	 \item \href{https://play.rust-lang.org/?version=stable&mode=debug&edition=2018&gist=0062381305db283f03f7511fcaabd1c2}
	       {(Playground)}
	\end{itemize}
 \end{itemize}
 {\scriptsize
\begin{lstlisting}[language=Rust,style=boxed,style=colouredRust]
if value < 1 {
    panic!("Guess value must be less than or equal to 100, got {}.",
           value);
} else if value > 100 {
    panic!("Guess value must be greater than or equal to 1, got {}.",
           value);
}\end{lstlisting}}
\end{frame}

\begin{frame}[containsverbatim]{Panic with \texttt{should\_panic} (8/8)}
\begin{itemize}
 \item 無事失敗する
       \begin{itemize}
	\item パニックの文言が表示
	\item 表示されるべき文言がnoteに表示
       \end{itemize}
\end{itemize}
\begin{commandline}
failures:

---- tests::greater_than_100 stdout ----
thread 'tests::greater_than_100' panicked at \
'Guess value must be greater than or equal to 1, got 200.', src/lib.rs:14:13
note: Run with `RUST_BACKTRACE=1` for a backtrace.
note: Panic did not include expected string \
'Guess value must be less than or equal to 100'
\end{commandline}
\end{frame}

\begin{frame}[containsverbatim]{Using \texttt{Result<T,E>} (1/3)}
 \begin{itemize}
  \item ここまで、テストに \verb|panic!| を使ってきた
  \item テストには \verb|Result<T, E>| も使える
	\begin{itemize}
	 \item 先程見た \verb|it_works()| 関数を \verb|Result| で書き換え
	 \item 成功時に \verb|Ok(())| を返し、失敗時に文言を含んだ \verb|Err| を返す
	 \item \href{https://play.rust-lang.org/?version=stable&mode=debug&edition=2018&gist=774593cbe99583e34456249c55b26116}
	       {(Playground)}
	\end{itemize}
 \end{itemize}
 {\scriptsize
\begin{lstlisting}[language=Rust,style=boxed,style=colouredRust]
#[cfg(test)]
mod tests {
    #[test]
    fn it_works() -> Result<(), String> {
        if 2 + 2 == 4 {
            Ok(())
        } else {
            Err(String::from("two plus two does not equal four"))
        }
    }
}\end{lstlisting}}
\end{frame}

\begin{frame}[containsverbatim]{Using \texttt{Result<T,E>} (2/3)}
 \begin{itemize}
  \item 出力
 \end{itemize}
\begin{commandline}
test tests::it_works ... ok

test result: ok. 1 passed; 0 failed; 0 ignored; 0 measured; 0 filtered out
\end{commandline}

\begin{itemize}
 \item (追試): $2 + 2$ を $2 + 3$ としたら以下のように
       \begin{itemize}
	\item \verb|assert_eq!| みたいな出力だが
	      \begin{itemize}
	       \item 左辺はエラー結果(非0)の値を意味している模様
	       \item 右辺は成功時の結果(0)
	      \end{itemize}
       \end{itemize}
\end{itemize}
\begin{commandline}
Error: "two plus two does not equal four"
thread 'tests::it_works' panicked at \
'assertion failed: `(left == right)`
  left: `1`,
 right: `0`: the test returned a termination value \
with a non-zero status code (1) which indicates a failure', src/libtest/lib.rs:337:5
\end{commandline}
\end{frame}

\begin{frame}[containsverbatim]{Using \texttt{Result<T,E>} (3/3)}
 \begin{itemize}
  \item この方法を使う場合 \verb|#[should_panic]| は使えない
  \item 替わりに \verb|Err| を返すようにしましょう
 \end{itemize}
\end{frame}

\end{document}

