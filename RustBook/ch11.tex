\documentclass[cjk,14pt,xcolor=dvipsnames,table,dvipdfmx,professional font,t,fragile]{beamer}
\usetheme{Copenhagen}
\usepackage{ulem}
\usepackage{tabularx}
\usepackage{fancybox}
\usepackage{fancyvrb}
\usepackage{float}
\usepackage{multicol}
\usepackage{minijs}
\usepackage{amsmath}
\usepackage{amssymb}
%\usepackage{minted}
\usepackage{ulem}
\usepackage{newtxtext}
\usepackage{listings, listings-rust}
\usepackage{hyperref}


\renewcommand{\familydefault}{\sfdefault}
\renewcommand{\kanjifamilydefault}{\gtdefault}
%
\hypersetup{
 pdfauthor={小林 克希},
 pdftitle={RustBook勉強会},
 pdfkeywords={Rust},
 pdfsubject={},
 pdfcreator={pLaTeX + dvipdfmx},
 pdflang={Japanese}
}
%
\newenvironment{commandline}%
{\VerbatimEnvironment
  \begin{Sbox}\begin{minipage}{0.9\hsize}\begin{fontsize}{8}{8} \color{white} \begin{BVerbatim}}%
{\end{BVerbatim}\end{fontsize}\end{minipage}\end{Sbox}
  \setlength{\fboxsep}{8pt}
% start on a new paragraph

\vspace{6pt}% skip before
\fcolorbox{white}{black}{\TheSbox}

\vspace{3pt}% skip after
}
%end of commandlinesmall

\begin{document}
\title{RustBook勉強会}
\subtitle{11. Testing}
\author{Katsuki Kobayashi}
\date{2019-02?}

\maketitle

\begin{frame}{予めお詫び}
 \begin{itemize}
  \item 発表者はC言語/ARMアセンブラのみ書ける
  \item ということで、他の言語とかさっぱりです
	\begin{itemize}
	 \item 「あの言語では〜〜」とかいうのは\\
	       全体に聞いてください
	 \item 不幸にも最近C++の本を読まされています\\
	       タスケテ
	\end{itemize}
  \item 社風:「テスト? なにそれ食べられるの?」
  \item 「世間一般ではこんな感じ」的なツッコミ歓迎
 \end{itemize}
\end{frame}

\begin{frame}{自動テストを書こう!}
 \begin{itemize}
  \item ダイクストラ先生曰く(1972年のエッセイ)
	\begin{itemize}
	 \item 	{\scriptsize
		``Program testing can be very effective way to show the presence of bugs,
		but it is hopelessly inadequate for showing their absence.''}
	 \item バグの存在を示す効果的な方法ではあるが、\\
	       無い事を示すには絶望的に不充分である
	\end{itemize}
  \item でも、「テストを頑張るべきではない」、\\
	という意味ではない
 \end{itemize}
\end{frame}

\begin{frame}{正確さ(correctness)}
 \begin{itemize}
  \item Rustは正確さについてかなり考えられている
	\begin{itemize}
	 \item が、やっぱり正確であることの証明は大変
	\end{itemize}
  \item Rustの型システムはこの重役を担っている
	\begin{itemize}
	 \item が、型だけで不正確を全部抑えられない
	\end{itemize}
  \item というわけで、Rustは言語レベルでの\\
	テストの自動化をサポートします
 \end{itemize}
\end{frame}

\begin{frame}{例: \texttt{add\_two()} (1/2)}
 \begin{itemize}
  \item \texttt{add\_two()}
	\begin{itemize}
	 \item 入力された値に2を加えて返す関数
	 \item 入出力はともに整数
	\end{itemize}
  \item コンパイルするとRustがチェック
	\begin{itemize}
	 \item 型の一致 (Stringを入れたらエラー)
	 \item borrow (変な参照をしたらエラー)
	\end{itemize}
  \item 関数の挙動はノーチェック
	\begin{itemize}
	 \item $10$ 足されようが $50$ 引かれようが\\
	       エラーにはならない
	\end{itemize}
  \item 「自動テスト」の出番
 \end{itemize}
\end{frame}

\begin{frame}{例: \texttt{add\_two()} (2/2)}
 \begin{itemize}
  \item テストの方法の例
	\begin{itemize}
	 \item $3$ を与えたら $5$ を返すとassertをする
	 \item このテストを、コードの変更の度に行なう
	 \item 正しい挙動をしている事を確認
	\end{itemize}
 \end{itemize}
\end{frame}

\begin{frame}{このチャプターは}
 \begin{itemize}
  \item テストは難しい
	\begin{itemize}
	 \item このチャプター
	\end{itemize}
 \end{itemize}
\end{frame}

\begin{frame}{How to Write Tests}
 \begin{itemize}
  \item Rustのテスト
	\begin{itemize}
	 \item コードが期待通り動くかを確認するテスト関数
	\end{itemize}
  \item テスト関数では一般に以下の3つのアクションを実行
	\begin{itemize}
	 \item 必要なデータや状態を準備する
	 \item テストしたいコードを実行する
	 \item 結果が期待通りかアサートする
	\end{itemize}
  \item 以降、Rustがテストのために提供しているものを紹介
	\begin{itemize}
	 \item \texttt{test} attribute
	 \item マクロ
	 \item \texttt{should\_panic} attribute、等
	\end{itemize}
 \end{itemize}
\end{frame}

\begin{frame}[containsverbatim]{The Anatomy of a Test Function}
 \begin{itemize}
  \item ぶっちゃけ、Rustのテストとは\\
	\texttt{test} attribute をつけた関数のこと
	\begin{itemize}
	 \item attribute: Chapt. 5で使った \texttt{derive} とか
	\end{itemize}
  \item test attributeの付け方: \\
	\hspace{2zw} \texttt{fn} の前に \verb|#[test]| を付ける
  \item \texttt{test} attributeを付けると
	\begin{itemize}
	 \item \texttt{cargo test}コマンド
	 \item テストランナーをビルドして実行
	 \item レポートを表示 (passes or fails)
	\end{itemize}
 \end{itemize}
\end{frame}

\begin{frame}[containsverbatim]{The Anatomy of a Test Function(1/6)}
 \begin{itemize}
  \item \verb|cargo new adder --lib| する
 \end{itemize}
 \begin{commandline}
% cargo new adder --lib
      Created library `adder` package
 \end{commandline}
 \begin{itemize}
  \item \texttt{src/libs.rs} が以下の内容で作成される
 \end{itemize}
 {\scriptsize
 \begin{lstlisting}[language=Rust,style=boxed,style=colouredRust]
#[cfg(test)]
mod tests {
    #[test]
    fn it_works() {
        assert_eq!(2 + 2, 4);
    }
}\end{lstlisting}}
 \begin{itemize}
  \item とりあえず \verb|#[cfg(test)]| と \verb|mod tests| の行は無視
 \end{itemize}
\end{frame}

\begin{frame}[containsverbatim]{The Anatomy of a Test Function(2/6)}
 \mbox{}
 {\scriptsize
 \begin{lstlisting}[language=Rust,style=boxed,style=colouredRust]
#[cfg(test)]
mod tests {
    #[test]
    fn it_works() {
        assert_eq!(2 + 2, 4);
    }
}\end{lstlisting}}
 \begin{itemize}
  \item \verb|#[test]|
	\begin{itemize}
	 \item テスト関数を示す
	 \item 共通処理等を書く非テスト関数も書けるように
	\end{itemize}
  \item \verb|assert_eq!(2 + 2, 4)|
	\begin{itemize}
	 \item $2 + 2$ が $4$ である事をアサート
	\end{itemize}
 \end{itemize}
\end{frame}

\begin{frame}[containsverbatim]{The Anatomy of a Test Function(3/6)}
 \begin{commandline}
% cargo test
   Compiling adder v0.1.0 (/home/rare/work/slides/RustBook/samples/adder)
    Finished dev [unoptimized + debuginfo] target(s) in 0.53s
     Running target/debug/deps/adder-1d0dfd1494bab5c6

running 1 test
test tests::it_works ... ok

test result: ok. 1 passed; 0 failed; 0 ignored; 0 measured; 0 filtered out

   Doc-tests adder

running 0 tests

test result: ok. 0 passed; 0 failed; 0 ignored; 0 measured; 0 filtered out
 \end{commandline}
 \begin{itemize}
  \item \href{https://play.integer32.com/?version=stable&mode=debug&edition=2018&gist=ce8ee04d2dcc2311a14efc8090ebda51}
	{PlayGround} でもテストできる……
  \item ignored, filteredについては次節で説明
  \item measuredについては現在はnightlyだけの機能
	\begin{itemize}
	 \item ベンチマークテスト用
	\end{itemize}
 \end{itemize}
\end{frame}

\begin{frame}[containsverbatim]{The Anatomy of a Test Function(4/6)}
 \begin{itemize}
  \item \verb|Doc-tests adder| という出力がある
	\begin{itemize}
	 \item ドキュメントテストの結果
	 \item APIのドキュメントのコーディング例もコンパイル/テスト可能
	 \item ドキュメントとコードの同期に役立つ!!
	\end{itemize}
  \item Chapt. 14 ``Documentation Comments'' でやるのでスルーします
 \end{itemize}
\end{frame}

\begin{frame}[containsverbatim]{The Anatomy of a Test Function(5/6)}
 \begin{itemize}
  \item テストの名前を変更
	\begin{itemize}
	 \item \verb|it_works()| を \verb|exploration()| にしてみる
	 \item  \href{https://play.integer32.com/?version=stable&mode=debug&edition=2018&gist=dc15fce8e988e1efd0a2dd8d8d87dd27}
		{PlayGround}
	 \item 無事に結果の表示に反映
	\end{itemize}
 \end{itemize}
 \begin{commandline}
running 1 test
test tests::exploration ... ok
 \end{commandline}
\end{frame}

\begin{frame}[containsverbatim]{The Anatomy of a Test Function(6/6)}
 \begin{itemize}
  \item 失敗するテストを入れてみる
	\begin{itemize}
	 \item \verb|panic!| マクロを使って \verb|another()| 関数を作る
	 \item \href{https://play.integer32.com/?version=stable&mode=debug&edition=2018&gist=71238afa6168ec8b26d4bf761d6404a6}
	       {PlayGround}
	\end{itemize}
 \end{itemize}
 {\scriptsize
 \begin{lstlisting}[language=Rust,style=boxed,style=colouredRust]
    #[test]
    fn another() {
        panic!("Make this test fail");
    }\end{lstlisting}}
 \begin{itemize}
  \item 無事失敗する (前回より2つ項目が増えてる)
 \end{itemize}
 \vspace*{-10pt}
 \begin{commandline}
test tests::another ... FAILED

failures:

---- tests::another stdout ----
thread 'tests::another' panicked at 'Make this test fail!', src/lib.rs:10:9

failures:
    tests::another
 \end{commandline}
\end{frame}

\begin{frame}[containsverbatim]{Checking Results with the assert! (1/)}
 \begin{itemize}
  \item \verb|assert!| マクロ
	\begin{itemize}
	 \item true: 何もしない
	 \item false: \verb|panic!| をコール
	\end{itemize}
  \item Chapt. 5の \texttt{Rectanble}構造体を例に
	\begin{itemize}
	 \item 	\href{https://play.integer32.com/?version=stable&mode=debug&edition=2018&gist=dc15fce8e988e1efd0a2dd8d8d87dd27}
		{PlayGround}
	\end{itemize}
 \end{itemize}
 {\scriptsize
 \begin{lstlisting}[language=Rust,style=boxed,style=colouredRust]
#[derive(Debug)]
pub struct Rectangle {
    length: u32,
    width: u32,
}

impl Rectangle {
    pub fn can_hold(&self, other: &Rectangle) -> bool {
        self.length > other.length && self.width > other.width
    }
}\end{lstlisting}}

\end{frame}

\begin{frame}[containsverbatim]{Checking Results with the assert! (2/)}
\mbox{}
 {\scriptsize
 \begin{lstlisting}[language=Rust,style=boxed,style=colouredRust]
#[derive(Debug)]
pub struct Rectangle {
    length: u32,
    width: u32,
}

impl Rectangle {
    pub fn can_hold(&self, other: &Rectangle) -> bool {
        self.length > other.length && self.width > other.width
    }
}\end{lstlisting}}
 \begin{itemize}
   \item \verb|can_hold()| メソッド
	 \begin{itemize}
	  \item ブール値を返す\\
		\hspace{3zw} $\rightarrow$ \verb|assert!| でのテストに最適!!
	  \item \verb|length| が $8$ , \verb|width| が $7$ の \verb|Rectangle| が\\
		\verb|length| が $5$ , \verb|width| が $1$ の \verb|Rectangle|
		をholdできる事をテスト
	 \end{itemize}
 \end{itemize}
\end{frame}

\begin{frame}[containsverbatim]{Checking Results with the assert! (3/)}
 \mbox{}
 {\scriptsize
 \begin{lstlisting}[language=Rust,style=boxed,style=colouredRust]
    #[test]
    fn larger_can_hold_smaller() {
        let larger = Rectangle { length: 8, width: 7 };
        let smaller = Rectangle { length: 5, width: 1 };

        assert!(larger.can_hold(&smaller));
    }\end{lstlisting}}
\begin{itemize}
 \item \href{https://play.integer32.com/?version=stable&mode=debug&edition=2018&gist=cdf6bb94826b7c2d369e3419ac251cac}
       {PlayGround}
\end{itemize}
\end{frame}

\end{document}

